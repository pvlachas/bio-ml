\documentclass[11pt,a4paper]{article}

\usepackage[utf8]{inputenc}
\usepackage{amsmath, amssymb, amsfonts}
\usepackage{geometry}
\geometry{margin=2.5cm}

\title{Step by Step Derivation of the Evidence Lower Bound}
\author{Tutorial Notes}
\date{}

\begin{document}
\maketitle

\section{Setup}
Let \(X\) be observed data. Let \(Z\) be latent variables. Let \(B\) be known covariates such as batch indicators. The generative model is
\[
p_\theta(X,Z \mid B) \;=\; p(Z)\, p_\theta(X \mid Z,B),
\]
with parameters \(\theta\). Our goal is to learn \(\theta\) by maximizing the conditional log evidence
\[
\log p_\theta(X \mid B) \;=\; \log \int p_\theta(X,Z \mid B)\, dZ.
\]
The integral is usually intractable when \(Z\) is high dimensional.

\section{Introduce a variational posterior}
Pick a tractable distribution \(q_\phi(Z \mid X,B)\) with parameters \(\phi\). Multiply and divide inside the integral by this distribution
\begin{align*}
\log p_\theta(X \mid B)
&= \log \int p_\theta(X,Z \mid B)\, dZ
= \log \int q_\phi(Z \mid X,B)\, \frac{p_\theta(X,Z \mid B)}{q_\phi(Z \mid X,B)}\, dZ \\
&= \log\, \mathbb{E}_{q_\phi(Z \mid X,B)}\!\left[ \frac{p_\theta(X,Z \mid B)}{q_\phi(Z \mid X,B)} \right].
\end{align*}

\section{Apply Jensen inequality}
The logarithm is concave. Jensen inequality gives
\begin{align*}
\log p_\theta(X \mid B)
&\ge \mathbb{E}_{q_\phi(Z \mid X,B)}\!\left[ \log \frac{p_\theta(X,Z \mid B)}{q_\phi(Z \mid X,B)} \right] \\
&= \mathbb{E}_{q_\phi}\!\left[ \log p_\theta(X,Z \mid B) \right] \;-\; \mathbb{E}_{q_\phi}\!\left[\log q_\phi(Z \mid X,B)\right].
\end{align*}
Define this expectation as the evidence lower bound
\[
\mathcal{L}_{\text{ELBO}}(\theta,\phi)
\;\equiv\;
\mathbb{E}_{q_\phi}\!\left[ \log p_\theta(X,Z \mid B) \right]
- \mathbb{E}_{q_\phi}\!\left[\log q_\phi(Z \mid X,B)\right].
\]

\section{Expand the model joint}
Use the factorization \(p_\theta(X,Z \mid B)=p(Z)\,p_\theta(X \mid Z,B)\)
\begin{align*}
\mathcal{L}_{\text{ELBO}}(\theta,\phi)
&= \mathbb{E}_{q_\phi}\!\left[ \log p_\theta(X \mid Z,B) \right]
+ \mathbb{E}_{q_\phi}\!\left[ \log p(Z) \right]
- \mathbb{E}_{q_\phi}\!\left[ \log q_\phi(Z \mid X,B) \right].
\end{align*}
Recognize the Kullback–Leibler divergence
\[
\mathrm{KL}\!\big(q_\phi(Z \mid X,B)\,\|\,p(Z)\big)
= \mathbb{E}_{q_\phi}\!\left[ \log q_\phi(Z \mid X,B) - \log p(Z) \right].
\]
Hence the ELBO can be written as
\[
\boxed{
\mathcal{L}_{\text{ELBO}}(\theta,\phi)
= \mathbb{E}_{q_\phi(Z \mid X,B)}\!\left[ \log p_\theta(X \mid Z,B) \right]
- \mathrm{KL}\!\big(q_\phi(Z \mid X,B)\,\|\,p(Z)\big).
}
\]

\section{Identity that links ELBO to the true evidence}
Start from the divergence between the variational and true posteriors
\[
\mathrm{KL}\!\big(q_\phi(Z \mid X,B)\,\|\,p_\theta(Z \mid X,B)\big)
= \mathbb{E}_{q_\phi}\!\left[\log q_\phi(Z \mid X,B) - \log p_\theta(Z \mid X,B)\right].
\]
Use Bayes rule \(p_\theta(Z \mid X,B)=\tfrac{p_\theta(X,Z \mid B)}{p_\theta(X \mid B)}\) to get
\begin{align*}
\mathrm{KL}&\!\big(q_\phi(Z \mid X,B)\,\|\,p_\theta(Z \mid X,B)\big) \\
&= \mathbb{E}_{q_\phi}\!\left[\log q_\phi(Z \mid X,B) - \log p_\theta(X,Z \mid B)\right] + \log p_\theta(X \mid B).
\end{align*}
Rearrange to obtain the key decomposition
\[
\boxed{
\log p_\theta(X \mid B)
= \mathcal{L}_{\text{ELBO}}(\theta,\phi)
+ \mathrm{KL}\!\big(q_\phi(Z \mid X,B)\,\|\,p_\theta(Z \mid X,B)\big).
}
\]
The divergence is nonnegative. Therefore the ELBO is a lower bound on the log evidence. Maximizing the ELBO tightens the bound and moves the variational posterior closer to the true posterior.

\section{Observed labels as an auxiliary head}
If the model also generates labels \(C\) from \(Z\) with \(p_\theta(C \mid Z)\) and labels are observed, the joint becomes
\[
p_\theta(X,C,Z \mid B) \;=\; p(Z)\, p_\theta(C \mid Z)\, p_\theta(X \mid Z,B).
\]
Repeat the same steps to obtain
\[
\boxed{
\mathcal{L}_{\text{ELBO}}(\theta,\phi)
= \mathbb{E}_{q_\phi}\!\Big[ \log p_\theta(X \mid Z,B) \;+\; \log p_\theta(C \mid Z) \Big]
- \mathrm{KL}\!\big(q_\phi(Z \mid X,B)\,\|\,p(Z)\big).
}
\]

\section{Practical estimation of the expectation}
The expectation with respect to \(q_\phi(Z \mid X,B)\) is computed by Monte Carlo. For a Gaussian encoder with diagonal covariance
\[
q_\phi(Z \mid X,B) = \mathcal{N}\!\big(\mu_\phi(X,B), \mathrm{diag}(\sigma_\phi^2(X,B))\big),
\]
use the reparameterization trick
\[
Z = \mu_\phi(X,B) + \sigma_\phi(X,B) \odot \epsilon, \quad \epsilon \sim \mathcal{N}(0,I).
\]
This representation isolates the randomness in \(\epsilon\) and allows gradients to flow through \(\mu_\phi\) and \(\sigma_\phi\). The KL term has a closed form when \(p(Z)=\mathcal{N}(0,I)\)
\[
\mathrm{KL}\!\big(\mathcal{N}(\mu,\mathrm{diag}(\sigma^2))\,\|\,\mathcal{N}(0,I)\big)
= \tfrac{1}{2}\sum_{k=1}^d \left( \mu_k^2 + \sigma_k^2 - \log \sigma_k^2 - 1 \right).
\]
The reconstruction term is the log likelihood under the decoder. For count data modeled with a Negative Binomial this is the corresponding log probability given the decoder outputs and dispersion.

\section{Mini batches and sums over data}
For independent data points \(\{x_i,b_i\}_{i=1}^N\), the total ELBO is a sum
\[
\mathcal{L}_{\text{ELBO}} = \sum_{i=1}^N
\Big(
\mathbb{E}_{q_\phi(z_i \mid x_i,b_i)}[\log p_\theta(x_i \mid z_i,b_i)]
- \mathrm{KL}(q_\phi(z_i \mid x_i,b_i)\,\|\,p(z_i))
\Big),
\]
possibly plus label terms. In practice one maximizes a stochastic estimate based on a mini batch and scales the reconstruction term to be an unbiased estimator of the full sum.

\section{Summary}
The ELBO arises by introducing a tractable posterior, applying Jensen, expanding the model joint, and recognizing a KL divergence. The bound decomposes into a reconstruction term and a regularization term. Maximizing the ELBO both increases the conditional evidence and reduces the gap to the true posterior.
\end{document}
